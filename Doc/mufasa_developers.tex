\documentclass[a4paper, 10pt]{report} % perhaps book?

% For images
\usepackage{graphicx}

% For URLs

\usepackage{url}

% For all our math joys
\usepackage{amsmath}

% For correct line-breaking.
\usepackage[english]{babel}

% Syntax highlighting.
% \usepackage{listings}

\usepackage{color}

\setlength{\parskip}{1.5ex}

\begin{document}
\title{Mufasa Developers Manual}
\author{Merlijn Wajer (Wizzup?) \and Raymond van Veneti\"{e} (mastaraymond) 
\and Benjamin J Land (BenLand100) \and Nielsie95}
\maketitle
\tableofcontents
% \lstset{language=Pascal}
% \definecolor{lightgray}{rgb}{0.9,0.9,0.9} 
% \definecolor{mycommentcol}{rgb}{0.2,0.2,0.8} 
% 
% \lstset{morecomment=[l][\color{mycommentcol}]{//},
% morecomment=[s][\color{mycommentcol}]{/*}{*/}}
% \lstset{backgroundcolor=\color{lightgray}}

\chapter{Foreword}

A word of thanks to the SRL Community\footnote{http://villavu.com/} and some of
it's members.
The following persons (in no particular order) have contributed in some way to
Simba, note that not all names correspond to the real name of a person, but
rather to their corresponding internet name.
Benjamin J Land (aka BenLand100), Nielsie95, Markus, Bullzeye95,
Yakman, Mixster, ss23, Nava2, Dgby714, Widget.

A special word of thanks goes to Frederic Hannes (Freddy1990) for maintaining
his program\footnote{SCAR, http://freddy1990.com}, for giving us inspiration
and generally for all the time he has put in SRL.

That's about it for the foreword, we hope this document will be of any use to
you, as reader.

Wizzup? and Raymond

\chapter{Introduction}

As Simba depends heavily on the MML and the MML can also actively be used in 
other languages, we will first discuss the MML, and then turn to Simba.

This is the Simba/MML (Mufasa Macro Library) documentation, aimed at
developers. The document has several different parts, each explaining
Simba/MML internals. We will first discuss the general structure of a
particular subject, and if necessary, spend a few sections on files
linked with the subject.

\chapter{MML}

The MML\footnote{Mufasa Macro Library} consists of several
modular\footnote{Modules are not completely self-contained}
classes / objects. Each of these classes strive to be
completely platform independent. We will look at each of these classes.
\footnote{
The last class, TMFiles, may be removed in the future, as it doesn't
perform any specific operations that are hard to do on other platforms or
operating systems.
}

\begin{itemize}
    \item   TIOManager
    \item   TMFinder
    \item   TMBitmaps
    \item   TMDTM
    \item   TMOCR
    \item   TMFiles
\end{itemize}

And finally, to bundle all these components into one class, the MML contains a
TMClient class, which simply initialises all the previously mentioned classes
when it is created, and destroys them when it is destroyed. It also allows
modules to access other modules, using the TMClient reference.

\section{TMClient}

\subsection{Introduction}

% Don't confuse with a client

\section{TIOManager}


\subsection{Introduction}

\subsection{TTarget}
\subsection{TRawTarget}
\subsection{TBitmapTarget}
\subsection{TWindow\_Abstract}
\subsection{TEIOS\_Client}
\subsection{TEIOS\_Target}
\subsection{TEIOS\_Controller}
\subsection{TTarget\_Exported}
\subsection{TIOManager\_Abstract}

\section{TMFinder}

\subsection{Introduction}

The TMFinder class is basically a large collection of different object
\footnote{Object being either a colour, bitmap or dtm} ``finding'' methods.
It has a reference to it's ``parent'' Client object, since it needs to have
access to TIOManager for retreiving client data, and access to managed bitmaps
and DTMs in TMBitmaps and TMDTM.


\subsection{Caching}
% ClientTPA + CachedWidth/CachedHeight.

\subsection{Colour Comparison Algorithms}

% CTS 0,1,2


\section{TMBitmaps}

\section{TMDTM}

\subsection{The DTM}

DTM stands for Deformable Template Model. \\

\emph{``DTM'' is the term used in SCAR. If it is actually a Deformable Template
Model is definately debateable; but for now we will stick to ``DTM''.} \\

A DTM is a relatively simple way of defining a relationship between several
points. Each of these points have a relative offset to each other, and each 
stores its own colour, tolerance, area size, and area shape. A DTM consists
of one \textbf{main point}, and several \textbf{sub-points}

+The main point's value is typically $ (0, 0) $, and all the
+sub point points are relative to the main point. "Point match" defines if
+a given location should or should \textbf{not} match.

The structure of a DTM looks like this:

%\begin{figure}[ht]
%    \includegraphics[scale=0.4]{Pics/DTM}
%    \caption{Structure of a DTM.}
%\end{figure}

Where each point in a DTM has a colour, tolerance, area size and area shape
entity. The main point's ``point'' is typically $ (0, 0) $, and all the
sub point points are arelative to the main point. ``Point Match'' defines
 if a point should match or should \textbf{Not} match.

The actual representation in Pascal is slightly different:


\begin{verbatim}
  pDTM = record
    l: Integer;
    p: TPointArray;
    c, t, asz, ash: TIntegerArray;
    bp: Array Of Boolean;
    n: String;
  end;
\end{verbatim}

\subsubsection{DTM Example}

If one was to create his own DTM, s/he would first have to
think of a useful DTM structure.

Say:
$$ MainPoint = (123, 456)  $$
$$ SubPoint_1 = (122, 460) $$
$$ SubPoint_2 = (120, 450) $$

We could then create the following pDTM structure:

\begin{verbatim}
	// Give dtm.p a length of three.
	// Mainpoint
	dtm.p[0] = Point(123, 456);

	// Subpoints
	dtm.p[1] = Point(122, 460)
	dtm.p[2] = Point(120, 450)
\end{verbatim}

Note that we do not include other variables, such as colour, tolerance, area
size and area shape; they should be handled in a similar manner.

However, this code is not very clear about the relation between the DTM's
 points. It would be better to write:

\begin{verbatim}
    // Give dtm.p a length of three.
    // Mainpoint
    dtm.p[0] = Point(0, 0);

    // Subpoints
    dtm.p[1] = Point(-1, 4)  // 122 - 123 = -1, 460 - 456 = 4
    dtm.p[2] = Point(-3, -6) // 120 - 123 = -3, 450 - 456 = -6
\end{verbatim}

As you can see it is perfectly valid to use negative points.

\subsubsection{Color and Tolerance}

The colour value of a point in a DTM is just a RGB integer value.
Black = 0, Red = 255, White = 16777215, etc.

The value tolerance decides if a colour is similar enough to the given
colour; if this is the case, we say that the colours matched.

With no area size and area shape specified\footnote{With area size set to zero
 and area shape specified as rectangle, the default} we say that a DTM matches if
for each point in the DTM, the colour at the relative point matches the colour
 in dtm with the given tolerance.

$$ \forall p \in P, \forall t \in Tol, \forall c \in Col : T(C(p), c) \leq t
 $$

With C() defining the colour at the given point, and T() defining the tolerance
between the two given colours.

\subsubsection{Area Size and Shape}

Area size and shape add that nifty extra functionality to DTM's.
\textbf{Area size} defines the area that should all match the colour
with the given tolerance. \\
\textbf{Area shape} is currently not implemented, mainly because
current aplications work well with rectangular shapes.

\subsection{How does TMDTM fit in?}

The TMDTM class is a DTM manager. It provides methods to add, store, load
and free DTM's. It has a few other features. One of its other features
is keeping track of what DTMs are unfreed. It can also, for example, help you
find a bug in your code, by printing out information of the DTM as it if used 
You can also give names to DTMs, which eases debugging even further.

If you try to access an invalid DTM, the MML will throw an exception.

\section{TMOCR}

\section{TMFiles}

\section{Finding algorithms}

\section{Portability to other languages}

Since it is near to impossible to simply import the MML classes, a library
called ``libmml'' is currently being written, which will offer a non-OOP 
wrapper.

\chapter{Simba - the GUI}

\section{Introduction}

Simba is a frontend to the MML. It allows one to develop and run scripts that 
use the MML. It features tabs, where each tab can not only hold a file, but also
run a script; you can run multiple scripts at once in Simba. \\

% XXX Move out of here?
Simba also features some tools to ease development. There is a function list
consisting of in-build functions, functions from includes and functions in your
script. There is also a component called ``Auto completion'', which shows all
possible options for your code. Another feature is ``Code hints'', which shows
the variables a specific function requires.
% XXX

\section{Window selecting, Colour picking and mouse position polling}
Simba internally uses an instance of TMClient, for window selecting, colour
picking and mouse position polling. If a script instance is started, the
currently selected window handle in Simba is passed to the script, and the script
then creates its own TMClient with the given window handle.

\section{Auto updaters}

Simba includes an auto updater for several components. Most importantly, Simba
itself. Simba compares its current version to an online one; on a different
thread. If the online version is greater than Simba's current version, it
downloads a new Simba executable and replaces the currently running Simba
executabke with the new one. On Windows this is done by renaming the old
Executable and deleting it on start. On Linux one can just replace the currently
running Simba.

Another auto updater in Simba is the font updater. It downloads the latest fonts
in the same manner as Simba itself, with versions.

\section{Extensions}

A great feature of Simba is its support for so called ``Extensions''.
This feature allows developers to quickly write an extension in a scripting
language, and then include it in Simba. An extension can vary from a simple
firewall to a rich bitmap editor. Currently, the Simba installer comes with a
SRL downloader and updater extension.

\section{Settings}

Simba contains a powerful component for Settings. It consists out of a XML
reader and writer; all Simba settings are stored in XML files. XML files can be
turned into Tree Views. Simba settings also support sandboxing, where the root
of a settings tree can be changed to a specific node. This can be used to
provide settings per script, without allowing the access to the other Simba
settings.

\section{Interpreting code}

See the next chapter titled ``Interpreters for Simba'' for more
information on Simba and interpreters.


% Loading/Saving
% Auto updating
% Settings?
% Code Completion?

\chapter{Interpreters for Simba}


\end{document}
